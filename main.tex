\documentclass{article}
\usepackage[utf8]{inputenc}
\usepackage{amsmath}
\usepackage{amsthm}
\usepackage{amssymb}
\usepackage{graphicx}
\title{Enumerative geometry on manifolds}
\author{Wei Jiang}
\date{August 2019}

\begin{document}

\theoremstyle{definition}
\newtheorem{df}{Def}[section]
\newtheorem{eg}[df]{Eg}
\newtheorem{thm}[df]{Thm}

\input{title/title.tex}
\tableofcontents
\newpage

\section{Introduction}

%%%%%%%%%%%%%%%%%%%%%%%%%%%%%%%%%%%%%%%%%%%%%%%%%%%%%%%%%%%%%%%%%%%%%%%%%%%%%%
\newpage
\section{Manifolds and Grassmannian}
 

When we try to describe curves on a 2-sphere, things become difficult. If we 
work in $\mathbb{R}^{3}$, we increse the complexity. If we work in $\mathbb{R}^{2}$,
indentifying some points on the sphere is inevitable, which over simplifies 
the question. To deal with a topological spaces like sphere are not easy, however 
we do have tools to describe such a space.

\begin{df}
An $n$-dimensional topological manifold is a topological space $X$ 
together with a set of open sets on $X$ ${U_{\alpha}}$ 
and maps $\phi_{\alpha} : U_{\alpha} \longrightarrow \mathbb{R}^{n} $
such that \\

(1) $X \subseteq \bigcup_{\alpha} U_{\alpha} $

(2) Each $\phi_{\alpha}$ is a homeomorphism from $U_{\alpha}$ to a open set 
$V_{\alpha} \subseteq \mathbb{R}^{n} $. \\

The pair $(U_{\alpha},\phi_{\alpha})$ is called a chart for $X$. The collection 
$\{(U_{\alpha},\phi_{\alpha})\}$ is called an atalas for $X$. \\

Each $\phi_{\alpha}$ defines a system of coordinates on $U_{\alpha}$, which is
the usual coordinates $(x_{1},x_{2},x_{3},...,x_{n})$ on 
$V_{\alpha} \subseteq \mathbb{R}^{n}$. Thiese are the local coordinates on $U_{\alpha}$.
Functions $f_{\alpha\beta} : V_{\beta}\longrightarrow V_{\alpha}$ defined by 
$f_{\alpha\beta} = \phi_{\alpha} \circ \phi_{\beta}^{-1}$are called the transition
 functions.

\end{df}

Now we can do calculations on manifolds use the local coordinates on $U_{\alpha}$, 
and indentify calculations with different systems of coordinates using transion functions.

\begin{eg}
    The 2-sphere $S^{2}$ is a manifold.pick north and south poles, $(0,0,1)$ 
    and $(0,0,-1)$ on $S^{2} \subset \mathbb{R}^{2}$. We have one-to-one correspondences
    $\phi_{1} : S^{2}- \{(0,0,1)\} \longrightarrow \mathbb{R}^{2} $ and 
    $\phi_{2} : S^{2}- \{(0,0,-1)\} \longrightarrow \mathbb{R}^{2}$ defined by\\
    
    \[
    \phi_{1}(x,y,z) = (\frac{x}{1-z},\frac{y}{1-z}) \qquad 
    \phi_{2}(x,y,z) = (\frac{x}{1+z},\frac{y}{1+z})
    \]

    With inverses

    \begin{align*}
    \phi_{1}^{-1}(x,y) & = {}(\frac{2x}{x^{2}+y^{2}+1},\frac{2y}{x^{2}+y^{2}+1},\frac{x^{2}+y^{2}-1}{x^{2}+y^{2}+1}) \\
    \phi_{1}^{-1}(x,y) & = {}(\frac{2x}{x^{2}+y^{2}+1},\frac{2y}{x^{2}+y^{2}+1},-\frac{x^{2}+y^{2}-1}{x^{2}+y^{2}+1})
    \end{align*}
 
    We can easly show these are homeomorphisms, so $S^{2}$ is a manifold.
\end{eg}


\begin{df}
A Grassmannian $Gr(k,n)$ over $\mathbb{C}$ is the space of all $k$-dimensional 
linear subspace of $\mathbb{C}^{n}$. 
\end{df}

\begin{eg}
$Gr(1,n)$ the projective spaces $\mathbb{P}^{n}$ .
\end{eg}

\begin{eg}
$Gr(2,4)$ the hyperplanes in $\mathbb{P}^{4}$.
\end{eg}

\begin{thm}
$Gr(k,n)$ is a manifold with dimension $k(n-k) $. 
\end{thm}
\begin{proof}

\end{proof}

%%%%%%%%%%%%%%%%%%%%%%%%%%%%%%%%%%%%%%%%%%%%%%%%%%%%%%%%%%%%%%%%%%%%%%%%%%%%%%
\newpage
\section{Transverse intersection and cohomology}

%%%%%%%%%%%%%%%%%%%%%%%%%%%%%%%%%%%%%%%%%%%%%%%%%%%%%%%%%%%%%%%%%%%%%%%%%%%%%%
\newpage
\section{Vector bundles and chern class}

%%%%%%%%%%%%%%%%%%%%%%%%%%%%%%%%%%%%%%%%%%%%%%%%%%%%%%%%%%%%%%%%%%%%%%%%%%%%%%
\newpage
\section{Applications}
\end{document}
