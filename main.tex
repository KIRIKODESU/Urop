\documentclass[12pt]{article}
\usepackage[utf8]{inputenc}
\usepackage{tikz-cd}
\usepackage{amsmath}
\usepackage{amsthm}
\usepackage{amssymb}
\usepackage{graphicx}
\usepackage{natbib}
\usepackage{adjustbox}
\usepackage{fullpage}
\setlength{\parskip}{1em}
\title{Enumerative geometry of lines}
\author{Wei Jiang}
\date{August 2019}
\newcommand{\BigWedge}{\mathord{\adjustbox{valign=B,totalheight=.6\baselineskip}{$\bigwedge$}}}
\newcommand{\rexact}[3]{ #1 \longrightarrow #2 \longrightarrow #3 \longrightarrow 0}

\begin{document}



\theoremstyle{definition}
\newtheorem{df}{Definition}[section]
\newtheorem{eg}[df]{Example}


\theoremstyle{plain}
\newtheorem{thm}[df]{Thmorem}
\newtheorem{lm}[df]{Lemma}
\newtheorem{prop}[df]{Proposition}
\newtheorem{cor}[df]{Corollary}
\input{title/title.tex}
\tableofcontents
\newpage

\section*{Introduction}
\newpage

\section{Grassmannian as projective variety}

How can we parametrized all lines in a projective space $\mathbb{P}^{n}$ ? We know a line $l$ in $\mathbb{P}^{n}$ corresponds to a 2-dimensional linear subspace of $\mathbb{A}^{n}$. So the the question transfers to parameterising linear subspaces of a vector space.



\begin{df}
    Let $V$ be a vector space of dimension $n$, define $G(k,n)$ to be the set of all linear $k$-subspaces of $V$.
\end{df}
 
\begin{eg}
    A first example of such a space is projective spaces $\mathbb{P}^{n}$ or they can be denoted as $G(1,n + 1)$. Lines in $\mathbb{P}^{n}$ form $G(2,n+1)$. 
\end{eg}

To avoid confusion, if $\Lambda$ is a k-dimensional subspace of a n-dimension vector space $V$, then we use $[\Lambda] \in G(k,n)$ to denote the point corresponds to the subspace.
\\

It can be shown that Grassmannian is a projective variety, so we can use all the tools in intersection theory to deal with it. The structure can be seen via a method called Plucker embedding.\\

Let $\Lambda \subset V$ be a k-dimensional subspace, then $\BigWedge^{k}\Lambda$ is a 1-dimensional subspace of $\BigWedge^{n}V$. More precisely, if $v_{1},...,v_{k}$ is a basis of $\Lambda$, then $\BigWedge^{k}\Lambda$ is the line spanned by $ v_{1}\wedge...\wedge v_{k}$. This corresponds to a point of $\mathbb{P}\BigWedge^{k}V$. \\

(By $\mathbb{P}V$ we mean the quotient of $V$ by the relation $v~w$ iff $v = \lambda w$. This is similar to the construction of projective space)\\

This gives a map of sets $G(k,n) \longrightarrow \mathbb{P}\BigWedge^{k}V \cong \mathbb{P}^{ {n \choose k} -1}$. This map is called Plucker embedding.
\begin{lm}
    This map  $G(k,n) \longrightarrow \mathbb{P}\BigWedge^{k}V \cong \mathbb{P}^{ {n \choose k} -1}$ is injective.
\end{lm}

\begin{proof}
    let $v_{1},...,v_{k}$ be a basis of $\Lambda \subset V$, we can extend it to a basis of $V$ by adding linear independent vectors $ u_{k+1},...u_{n}$. Let $a =  v_{1}\wedge...\wedge v_{k}$, then $\forall v \in V$, $v \wedge a = 0$ if and only if $ (b_{1}v_{1}+...b_{n}u_{n})\wedge a = 0$ iff  $b_{k+1}u_{k+1}\wedge a + ... +  b_{n}u_{n}\wedge a =0 $ iff $ v \in \Lambda$ so $ a$ determines $\Lambda$.
\end{proof}

So $G(k,n) $ is isomorphic to its image in $\mathbb{P}\BigWedge^{k}V$. Lets call its image $G$. It left to show that its image is the common zero locus of some homogeneous polynomials. We can show this by express our k-subspace as a matrix.\\


Let ${e_1,e_2,...,e_n}$ is a basis for $V$,we can identify $V$ as $k^{n}$ then any $k$-vector space is the span of $k$ linear independent vectors in this basis. We can write them as a matrix:
$$
\begin{pmatrix}
    a_{1,1}& a_{1,2} & \dots & a_{1,n}\\
    a_{2,1}& a_{2,2} & \dots & a_{2,n}\\
    \vdots & \vdots & \ddots & \vdots \\
    a_{k,1}& a_{k,2} & \dots & a_{k,n}    
\end{pmatrix}
$$



However, just like coordinates of points in a projective space, the matrix $A$ is not unique. Since we can multiply on the left any invertible $k \times k$ matrix $\Omega$ without changing the row spaces, because the rows in $\Omega A$ are linear combinations of the rows in $A$.  
In this setting, $\BigWedge^{k}V$ is given by the set


\[ \{e_{i1}\wedge...\wedge e_{ik}\}_{1\leq i1<...<ik\leq n}\]


After the Plucker embedding, this matrix get sent to the wedge product of row vectors:


\[ v_{1}\wedge...\wedge v_{k} = \sum_{1\leq i1<...<ik\leq n} D_{i1,...,ik} e_{i1}\wedge...\wedge e_{ik}   \]

Where $D_{i1,...,ik}$ is the determinant of k minors of the matrix $A$. However, just like coordinates of points in a projective space, the matrix $A$ is not unique. Since we can multiply on the left any invertible $k \times k$ matrix $\Omega$ without changing the row spaces, because the rows in $\Omega A$ are linear combinations of the rows in $A$.If we choose another basis of $\Lambda$, then the result is multiply by the determinant of $\Omega$. As $\Omega$ invertible, the determinant is never $0$, so the product is always the same point in $\mathbb{P}\BigWedge^{k}V$.\\

Now we try to find these homogeneous polynomials that defines $Im(G(k,n))$. For any k-subspace, we can find k linearly independent vectors that span it. So $\omega \in \mathbb{P}\BigWedge^{k}V$ is in the image if and only if there exits $v_{1},...,v_{k}$(linearly independent) such that $\omega = v_{1}\wedge ... \wedge v_{k}$.

\begin{lm}
    $\omega \in \mathbb{P}\BigWedge^{k}V$ such that $\omega = v_{1}\wedge ... \wedge v_{k}$ if and only if the map:

    \[ V \xrightarrow{\wedge \omega} \mathbb{P}\BigWedge^{k+1}V\]

    has kernel of dimension at least k.
\end{lm}
    
\begin{proof}
    $(\Rightarrow)$ Trivial as ${v_{i}}$ is basis of the kernel.
    $(\Leftarrow)$ If the kernel has dimension at least k, then there exits $v_{1},...,v_{k}$(linearly independent), such that $v_{i}\wedge \omega = 0, 1\leq i \leq k$. \\
    Lets start from $v_{1}$, if $v_{i}\wedge \omega = 0$, then there exists $\omega' \in \mathbb{P}\BigWedge^{k-1}V$ such that $\omega = v_{0}\wedge \omega'$. Repeat this process, we can see that $\omega = v_{1}\wedge ... \wedge v_{k}$.
    
\end{proof}

Then the image can be expressed as:
 \[G = \{\omega \in \mathbb{P}\BigWedge^{k}V \ | \ \text{rank}(V \xrightarrow{\wedge \omega} \mathbb{P}\BigWedge^{k+1}V) \leq n-k  \} \]

This can be interpreted as the zero locus of degree $(n-k+1)$ homogeneous polynomials that are determinant of $(n-k+1)$ minors of the map $\wedge \omega: V \rightarrow \mathbb{P}\BigWedge^{k+1}V$ written as a matrix. So we can conclude that $G(k,n)$ is a variety.



\section{From group of cycles to chow ring}

Now we move on to these geometrical conditions that our lines should satisfy. Not all conditions are considered, as some of them are less interesting and others cannot be described by the language of variety. A suitable geometrical condition is one that corresponds to subvariety of a Grassmannian. Then the question of enumerating lines satisfy these condition be comes find 0-dimension intersections of of subvarieties of Grassmannian. However, Intersections of some subvarieties are hard to compute, and the dimension of intersections may vary when subvarieties intersect in different positions.As we can see, we only need the number of points in the intersections, we can solve the first problem by consider another set of subvarieties with same number of points in their intersections but easier to deal with. The second problem usually happens when our subvarieties has common component or they are tangent to each others, so we can solve this by only consider these subvarieties intersect generally. In this section, we will use the language of chow ring to described these two methods.

similar to the cup product of homology groups of cycle classes used in topology. We define a group of cycles then we define a suitable product.

\begin{df}
    Let $X$ be an variety. The group of cycles of $X$, denoted as $Z(X)$, is the free abelian group generated by the set of subvarieties.
\end{df}

This group $Z(X)$ is graded by dimensions, we use $Z_{k}(X)$ to denote the group of cycles that are formal linear sums of subvarieties of dimension k, these cycles are called k-cycles. So $Z(X) = \bigoplus _{k}Z_{k}(X)$. 

A cycle $Z= \sum n_{i}Y_{i}$, where the $Y_{i}$ are subvarieties, is effective if the coefficients $n_{i}$ are all nonnegative. A divisor is an (n-1)-cycle on a pure n-dimensional variety.

Now we show that to any subvarieties we can associate an effective cycle $\langle Y \rangle$ : If $Y \subset X $ is a subvariety with irreducible component $Y_{1}\dots Y_{m}$. Because the coordinate ring of $X$ is Noetherian, so is the coordinate ring of $Y$. We can localize $\mathcal{O}_{Y}$ at each $Y_{i}$, the result $\mathcal{O}_{Y,Y_{i}}$ is still Noetherian so it has a finite composition series. By Jordan-Holder theorem, the composition series has a well-defined length $l_{i}$, then define $\langle Y \rangle = \sum l_{i}Y_{i}$. $l_{i}$ is called the multiplicity if $Y$ along the irreducible component $Y_{i}$. 

Similar to homology groups of topological spaces, we consider cycles up to some equivalence relations. The idea for this relation is that two cycles $C_{1}, C_{2}$ are equivalent if there is a cycle on $\mathbb{P}^{1}\times X$ whose restriction to ${t_{1}} \times X$ is $C_{1}$ and restriction to ${t_{2}} \times X$ is $C_{2}$. Formally:

\begin{df}
    Let Rat($X$) $\subset$ $Z(X)$ be the group generated by difference of the form 
    \[ \langle \Phi \bigcap ({t_{1}} \times X) \rangle - \langle \Phi \bigcap ({t_{2}} \times X)\rangle \]

    where $t_{1},t_{2} \in \mathbb{P}^{1}$ and $\Phi$ is a subvariety of $\mathbb{P}^{1}\times X$ not contained in any fiber ${t} \times X$. We say that two cycles are rationally equivalent if their difference is in Rat($X$), and say two subvarieties are rationally equivalent if their associate cycles are rationally equivalent.
\end{df}

\begin{df}
    The Chow group of $X$ is the quotient 
    \[ A(X) = Z(X) / \text{Rat}(X)\]

the group of rational equivalence classes of cycles on $X$. If $Y \in Z(X) $ is a cycle, we write $[Y] \in A(X)$ for its equivalence class. When $Y \subset X $ is a subvariety, we still use $[Y]$ to denote the class of the cycle $\langle Y \rangle$ associate to $Y$.
\end{df}

Good news is Chow groups are also graded by dimension.

\begin{thm}
    If $X$ is a variety then the Chow group of $X$ is graded by dimension.
     \[ A(X) = \bigoplus A_{k}(X)\]

    with $A_{k}(X)$ the group of rational equivalence classes of k0cycles.
\end{thm}

\begin{proof}
    Let $\Phi \subset \mathbb{P}^{1} \times X $ is an irreducible variety not contained in a fiber over $X$. Consider in an affine open set $\Phi \bigcap(\mathbb{A}^{1} \times X)$ of $\Phi$, the subvariety $\Phi \bigcap ({t_{1}} \times X)$ is defined by the vanishing locus of $t - t_{1}$. So the component of this intersection are all of codimension $1$ in $\Phi$ ( this is by principle ideal theorem), and so is for $\Phi \bigcap ({t_{2}} \times X)$. As a result all varieties in the same class have the same dimension.
\end{proof}

When $X$ is equidimensional we can define the codimention of a subvariety $Y \subset X$ by dim $X$ - dim $Y$. So we can grade the Chow group by codimension. We use $A^{k}(X)$ to denote the group $A_{\text{x}-k}(X)$. The reason of this notation is that when we defined a product on this group, it becomes a graded ring in this notation. (This notation sometimes does not make sense if $X$ is singular. In our case, $X$ is always smooth so there is noting to worry about.)\\

[transverse intersections to be added]
\begin{thm}
    If $X$ is a smooth quasi-projective variety, then there is a unique product structure on $A(X)$ satisfying:\\
    If two subvarieties $A,B$ of $X$ are generically transverse, then 
    \[ [A][B] = [A \cap B ]\]
    This product makes
    \[   A(X) = \bigoplus^{\text{dim}X}_{k=0}A^{k}(X)\]
    into an associative, commutative ring, graded by codimension, called the Chow ring.
\end{thm}

\begin{thm}
    (Moving lemma). Let $X$ be a smooth quasi-projective vateiry.\\
    (a) For every $\alpha, \beta \in A(X)$ there are generically transverse cycles $A, B \in Z(X)$ with $[A] = \alpha$ and $[B] = \beta$.
    (b) The class $[A\cap B] $is independent of the choice of such cycles $A$ and $B$.
\end{thm}

\begin{proof}
    (a)\cite{3264} Appendix A.
    (b)\cite{fulton} 
\end{proof}



Computation of Chow rings

In this section we will introduce some techniques on computation of Chow rings.
by fundamental class, 

\begin{df}
    Let $X$ be a variety, then the fundamental class of $X$ is $[X] \in A(X)$. 
\end{df}

\begin{thm}
    Let $X$ be an irreducible variety of dimension n, then $A^{n}(X) \cong \mathbb{Z}$ and is generated by the fundamental class of $X$.
\end{thm}

For affine space

\begin{eg}(\cite{3264}, proposition 1.13)
    $A(\mathbb{A}^{n}) = \mathbb{Z} \cdot [\mathbb{A}^{n}].$

\end{eg}

similar to homology groups, we can computes Chow rings by component of the variety using Mayer-Vietoris sequence and excision.

\begin{thm}
    Let $X$ be a variety.

    (a) (Mayer-Vietoris) If $X_{1}$, $X_{2}$ are subvarieties of $X$, then there is a right exact sequence
    \[\rexact{A(X_{1}\cap X_{2})}{A(X_{1})\oplus A(X_{2})}{A(X_{1}\cup X_{2})} \]

    (b)(excision) If $Y \subset X$ is a subvariety and $U = X \setminus Y$, then the inclusion and restriction maps of cycles gives a right exact sequence
    \[\rexact{A(Y)}{A(X)}{A(U)} \]
    If $X$ is smooth, then the map $A(X) \longrightarrow A(U)$ is a ring homomorphism.
\end{thm}

\begin{proof}
    \cite{3264}proposition 1.14.
\end{proof}


\begin{cor}
    If $U \subset \mathbb{A}^{n}$ is a nonempty open set, then $A(U) = A_{n}(U) = \mathbb{Z} \cdot [U]$
\end{cor}

Affine stratifications 

The homology groups of a topological space are not easy to directly compute in general, however, when the topological space admits a cellular decomposition then the calculation is much easier.

Similarly idea applied in our case, if a variety $X$ admits an affine stratification( decomposition into open subsets of Affine space), then the calculation of Chow ring would be straight forward. And this is the main technique we used to compute the Chow rings of Grassmannian.

\begin{df}
    A variety is stratified by a finite collection of irreducible locally closed subset $U_{i}$ if $X$ is a disjoint union of the $U_{i}$ and the closure of any $U_{i}$ is a disjoin union of $U_{j}$ for some $U_{j}$. 

    The sets $U_{i}$ are called strata of stratification, while the closure $Y_{i} := \overline{U_{i}}$ are called closed strata.

\end{df}
 
The stratification can also be given by closed strata:
 \[ U_{i} = Y_{i} \setminus \bigcup_{Y_{j}\subsetneq Y_{i}} Y_{j}\]

 \begin{df}
     A stratification of $X$ with strata $U_{i}$ is:\\
     Affine if each open strata is isomorphic to $\mathbb{A}^{k}$ for some k.\\
     Quasi-affine if each $U_{i}$ is isomorphic to an open subset of $\mathbb{A}^{k}$ for some k.

 \end{df}


\begin{eg}
   Lets consider projective space $\mathbb{P}^{n}$. There is a complete flag of subspaces $\mathbb{P}^{0}\subset \mathbb{P}^{1}\subset \dots \subset \mathbb{P}^{n}$. This gives a stratification of $\mathbb{P}^{n}$, the closed strata are the $\mathbb{P}^{i}$ and the open strata are affine spaces $U_{i}:= \mathbb{P}^{i}\setminus \mathbb{P}^{i-1} \cong \mathbb{A}^{i}$.
\end{eg}

\begin{prop}
    If a variety has a quasi-affine stratification, then $A(X)$ is generated by the classes of closed strata.
\end{prop}

\begin{proof}
    
\end{proof}

In general the classes of the strata in a quasi-affine stratification of a variety $X$ may be 0 in $A(X)$. For example, the affine line $\mathbb{a}^{1}$, with $A(\mathbb{A}^{1}) = \mathbb{Z}$, also has a quasi-affine stratification consisting of a single point and its complement, the class of a point is 0. However, in the case of an affine stratification, the classes are not only nonzero, they are independent as well.

\begin{thm}
    (\cite{totaro_2014} 2014). The classes of strata in an affine stratification of a variety $X$ form a basis of $A(X)$.
\end{thm}

This is a extremely powerful theorem which plays a central role in our computation in the next section. We omit the proof as it is out of the scope of this project.











\section{The Chow ring of $\mathbb{G}(1,3)$}

Now we would do a actual computation of the Chow ring of $\mathbb{G}(1,3)$. The idea is to build an affine stratification of $\mathbb{G}(1,3)$,this is done by classifying lines in $\mathbb{P}^{3}$.

By previous section, $\mathbb{P}^{3}$ admits an affine stratification by a complete flag $\mathcal{V}:= {p}\subset L \subset H \subset \mathbb{P}^{3}$. If we pick a specific flag. We can classifying line in $\mathbb{P}^{3}$ by the dimension of intersection with component of this flag. 

There are two key observations: 

(a) the dimension of the intersection of a line with any variety can only be 0 or 1.

(b) if the dimension of intersection of a variety with a component $V$ of the flag is k, then the dimension of intersection of that variety with any component containing $V$ is at least k.

With this observation and if we label the component of the flag by their dimensions. Then the set of  pair of number $\{(a,b) \ | \ 0 \leq a < b \leq 3\} $ can classify all possible dimension of intersection of component of $\mathcal{V}$ with a line $l$, where $a$ is the smallest number that $l$ intersect with the $a$-plane in a point and $b$ is the smallest number that $l$ is contained in the $b$-plane. All possible pairs are $ (0,1), (0,2), (0,3), (1,2), (1,3), (2,3)$. And these are called Schubert cells, and their closure are called Schubert cycles.

However, the bad news is that in the actual intersection theory, a rather peculiar notation is used. The notation for Schubert cycles is $\Sigma_{a,b} $ which corresponds to $\overline{(2-a, 3-b)}$ in our notation. So they stand for the closure of set of lines meeting the (2-a)-plane of $\mathcal{V}$ in a point and the (3-b)-plane of $\mathcal{V}$ in a line. The reason why people use this notation is that it is more convenient for Schubert cycles in general. And there is a bonus fact that we will not proof: the codimension of $\Sigma_{a,b} $ is $a + b$.

Here are the list of Schubert cycles of $\mathbb{G}(1,3)$ :
\begin{align*}
    \Sigma_{0,0} & =  \mathbb{G}(1,3)\\
    \Sigma_{1,0} & =   \{\Lambda \ | \ \Lambda \cap L \neq \emptyset\}  \\
    \Sigma_{2,0} & =  \{\Lambda \ | \ p \in \Lambda \}   \\
   \Sigma_{1,1} & =  \{\Lambda \ | \ \Lambda \subset H \}   \\
   \Sigma_{2,1} & =  \{\Lambda \ | \ p \in \Lambda \subset H\}\\
   \Sigma_{2,2} & =  \{\Lambda \ | \ \Lambda = L \}
\end{align*}


The cycles are apparently varieties, and in fact, we can show they are irreducible. Lets take $Sigma_{1,0}$ as an example, it can be viewed as the image of :
\[ \Gamma = \{(\Lambda, t) \in \mathbb{G}(1,3) \times L \ | \ t \in \Lambda \}\]

Consider the fibers along projection to $L$, they can be viewed as all lines passing through a point in $\mathbb{A}^{3}$, so they are isomorphic to $\mathbb{P}^{2}$. As all fibers are irreducible and isomorphic, $\Gamma$ is irreducible, so is $\Sigma_{1,0}$.

There are inclusions:
$$
\begin{tikzcd}
& & \Sigma_{2,0} \arrow[hookrightarrow]{dr} & & \\
\{L\} = \Sigma_{2,2} \arrow[hookrightarrow]{r} &\Sigma_{2,1} \arrow[hookrightarrow]{ur} \arrow[hookrightarrow]{dr} & & \Sigma_{1,0} \arrow[hookrightarrow]{r} &  \mathbb{G}(1,3) \\
& & \Sigma_{1,1} \arrow[hookrightarrow]{ur} & &
\end{tikzcd}.
$$

Now we left to show that this induce a affine stratification of $\mathbb{G}(1,3)$. 

Lets first consider  $U_{0,0} := \Sigma_{0,0} \setminus \Sigma_{1,0}$, this is the set of lines that does not intersect with $L$. We assume that $L = \{ x_{0} = x_{1}= 0\}$, then $U_{0,0}$ is the row space of

$$
\begin{pmatrix}
    1& 0 & a & b\\
    0& 1 & c & d
        
\end{pmatrix}
$$

Where $a,b,c,d $ has no constrains. So $U_{0,0}$ is isomorphic to $\mathbb{A}^{4}$.

Next lets consider $U_{1,0} := \Sigma_{1,0} \setminus( \Sigma_{2,0}\cup \Sigma_{1,1})$, assume the point is  $p = [0:0:0:1] \in L \subset \mathbb{P}^{3}$ and the plane is $H = Z(x_{0})$. So we want those lines that are not contained in $H$ and do not contain p. So element of the form $[0:0:1:e]$ is inside $U_{1,0}$.(note the third coordinate cannot be zero, otherwise it contains $p$.) So $U_{1,0}$ is the row space of

$$
\begin{pmatrix}
    1& f & 0 & g\\
    0& 0 & 1 & e
        
\end{pmatrix}
$$

Similarly it is isomorphic to $\mathbb{A}^{3}$.




























\newpage
A projective space can be covered by Zariski open subsets isomorphic to affine space. So we can define a local coordinates on projective space. Similarly, we can cover a Grassmannian $G = G(k,n)$ by Zariski open subsets as well. Too see this, fix an $(n-k)$-dimensional subspace $\Gamma \subset V$, and let $U_{\Gamma}$ be the subset of $k$-subspaces that do not meet $\Gamma$:
\[U_{\Gamma} = \{ L\ \in\ G \ | \ L\ \cap\ \Gamma = \emptyset \}
\]


Lets show it is Zariski open: let ${w_1,w_2,\dots ,w_{n-k} }$ be a basis of $\Gamma$ and let
 $\gamma = w_1 \wedge \dots \wedge w_{n-k}$then we have :

\[
U_{\Gamma} = \{ [\omega]   \in  G  \subset   \mathbb{P}(\BigWedge^k V) \ | \ \omega \wedge  \gamma \neq 0 \}
\]



%%%%%%%%%%%%%%%%%%%%%%%%%%%%%%%%%%%%%%%%%%%%%%%%%%%%%%%%%%%%%%%%%%%%%%%%%%%%%%
\newpage

\section{chern classes}

\bibliographystyle{plain}
\bibliography{ref}


\end{document}
