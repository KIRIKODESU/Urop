\documentclass{article} 
\usepackage[utf8]{inputenc}
\usepackage{amsmath}
\usepackage{amsthm}
\usepackage{amssymb}
\usepackage{graphicx}
\title{Enumerative geometry on manifolds}
\author{Wei Jiang}
\date{August 2019}

\begin{document}

\theoremstyle{definition}
\newtheorem{df}{Def}[section]
\newtheorem{eg}[df]{Eg}

\theoremstyle{plain}
\newtheorem{thm}[df]{Thm}
\newtheorem{lm}[df]{Lem}
\input{title/title.tex}
\tableofcontents
\newpage

\section{Introduction}

%%%%%%%%%%%%%%%%%%%%%%%%%%%%%%%%%%%%%%%%%%%%%%%%%%%%%%%%%%%%%%%%%%%%%%%%%%%%%%
\newpage
\section{Manifolds and Grassmannian}
 

When we try to describe curves on a 2-sphere, things become difficult. If we 
work in $\mathbb{R}^{3}$, we increse the complexity. If we work in $\mathbb{R}^{2}$,
indentifying some points on the sphere is inevitable, which over simplifies 
the question. To deal with a topological spaces like sphere are not easy, however 
we do have tools to describe such a space.

\begin{df}
An $n$-dimensional topological manifold is a topological space $X$
together with a set of open sets on $X$ ${U_{\alpha}}$ 
and maps $\phi_{\alpha} : U_{\alpha} \longrightarrow \mathbb{R}^{n} $
such that \\

(1) $X \subseteq \bigcup_{\alpha} U_{\alpha} $

(2) Each $\phi_{\alpha}$ is a homeomorphism from $U_{\alpha}$ to a open set 
$V_{\alpha} \subseteq \mathbb{R}^{n} $. \\

The pair $(U_{\alpha},\phi_{\alpha})$ is called a chart for $X$. The collection 
$\{(U_{\alpha},\phi_{\alpha})\}$ is called an atalas for $X$. \\

Each $\phi_{\alpha}$ defines a system of coordinates on $U_{\alpha}$, which is
the usual coordinates $(x_{1},x_{2},x_{3},...,x_{n})$ on 
$V_{\alpha} \subseteq \mathbb{R}^{n}$. Thiese are the local coordinates on $U_{\alpha}$.
Functions $f_{\alpha\beta} : V_{\beta}\longrightarrow V_{\alpha}$ defined by 
$f_{\alpha\beta} = \phi_{\alpha} \circ \phi_{\beta}^{-1}$are called the transition
 functions.

\end{df}

Now we can do calculations on manifolds use the local coordinates on $U_{\alpha}$, 
and indentify calculations with different systems of coordinates using transion functions.

\begin{eg}
    The 2-sphere $S^{2}$ is a manifold.pick north and south poles,$(0,0,1)$ 
    and $(0,0,-1)$ on $S^{2} \subset \mathbb{R}^{2}$. We have one-to-one correspondences
    $\phi_{1} : S^{2}- \{(0,0,1)\} \longrightarrow \mathbb{R}^{2} $ and 
    $\phi_{2} : S^{2}- \{(0,0,-1)\} \longrightarrow \mathbb{R}^{2}$ defined by\\
    
    \[
    \phi_{1}(x,y,z) = (\frac{x}{1-z},\frac{y}{1-z}) \qquad 
    \phi_{2}(x,y,z) = (\frac{x}{1+z},\frac{y}{1+z})1
    \]

    With inverses
    
    \begin{align*}
    \phi_{1}^{-1}(x,y) & = {}(\frac{2x}{x^{2}+y^{2}+1},\frac{2y}{x^{2}+y^{2}+1},\frac{x^{2}+y^{2}-1}{x^{2}+y^{2}+1}) \\
    \phi_{2}^{-1}(x,y) & = {}(\frac{2x}{x^{2}+y^{2}+1},\frac{2y}{x^{2}+y^{2}+1},-\frac{x^{2}+y^{2}-1}{x^{2}+y^{2}+1})
    \end{align*}
 
    We can easly show these are homeomorphisms, so $S^{2}$ is a manifold.
\end{eg}
%example of projective space

%definition of submanifold

To count the number of certain geometric varieties in a topological space, we need to have a way to describe 
all such varieties. We first consider linear ones.
\begin{df}
A Grassmannian $Gr(k,n)$ over $\mathbb{C}$ is the space of all $k$-dimensional 
linear subspace of $\mathbb{C}^{n}$. 
\end{df}

\begin{eg}
$Gr(1,n)$ the projective spaces $\mathbb{P}^{n}$ .
\end{eg}

\begin{eg}
$Gr(2,4)$ the hyperplanes in $\mathbb{P}^{4}$.
\end{eg}

\begin{thm}
$Gr(k,n)$ is a manifold with dimension $k(n-k) $. 
\end{thm}
\begin{proof}

\end{proof}
Then we consider the case when our varieties has degree $> 1$.

\begin{df}
    Let $X$ be a complex manifold with a atalas $\{U_{\alpha}\}$. If we have holomorphic
    functions $g_{\alpha \beta}$ defined on each ${U_{\alpha}\cap U_{\beta}}$ such that 
    $g_{\alpha\alpha} = 1$ and $g_{\alpha \gamma} = g_{\alpha \beta} g_{\beta \gamma}$ 
    on  each $U_{\alpha}\cup U_{\beta}\cap U_{\gamma} \neq \emptyset$, then we call 
    ${g_{\alpha\beta}}$ transitions subordinate to $\{U_{\alpha}\}$.

\end{df}

\begin{df}
    Let $g_{\alpha \beta }$ and $g_{\alpha \beta}^{'}$ be transition funtions subordinate
    to the open cover ${U_{|alpha}}$. $g\sim g^{'} $ if there are nowhere vanishing holomorphic functions
    $f_{\alpha}$ on ${U_{\alpha}}$ such that $g_{\alpha \beta } = f_{\alpha} g_{\alpha \beta}^{'} f_{\beta}^{-1}$.
    The set of equivalence classes is called the set of line bundles which can be trivialised over $\{U_{\alpha}\}$

    
\end{df}

\begin{df}
    A line bundle on a topological manifold $X$ with open cover $\{U_{\alpha}\}$ is a line bundle which can 
    be trivialised over some open covering ${U_{\alpha}}$. A line bundle $L$ trivialised over ${U_{\alpha}}$
    is equivalent to a line bundle $L^{'}$ trivialised over ${U_{\beta}^{'}}$ if they are equivalent
    after restricting all transition functions to the sets of open cover $\{U_{\alpha}\cap U_{\beta}^{'}\}$  
    
\end{df}

If we restricting to just complex manifold there is another way 
to view line bundles from a more geometric aspect. \\

Given transition functions $g_{\alpha\beta}$ subordinate to $\{ U_{\alpha}\}$.We build a new complex manifold
$E$ of dimension $ n +1$, where $n = \text{dim} X$. Then $E$ can be obtained
from the cover $\{U_{\alpha} \times \mathbb{C}\}$ by indentifying $(p,v_{\alpha}) \in 
U_{\alpha} \times \mathbb{C}$ and $(q,v_{\beta}) \in U_{\beta} \times \mathbb{C}$
 if $p = q$ and $v_{\alpha} = g_{\alpha\beta}(p) \cdot v_{\beta}$.
From this relations we define charts on $E$:

\begin{align*}
    \phi_{\alpha} \times 1 : U_{\alpha} \times \mathbb{C} \longrightarrow \mathbb{C}^{n} \times \mathbb{C} = \mathbb{C}^{n+1} 
\end{align*}

 
%%%%%%%%%%%%%%%%%%%%%%%%%%%%%%%%%%%%%%%%%%%%%%%%%%%%%%%%%%%%%%%%%%%%%%%%%%%%%%
\newpage
\section{Transverse intersection and cohomology}
Using Grassmannian and line bundles, we can describe curves or other algebraic varieties 
in a certain space( Eg. $\mathbb{C}^{n}$). Some of the conditions we used to enumerate, such as intersection
with a curve or containment in some hypersurfaces, are just some submanifolds of our parameter space.

\begin{eg}
    Conics in $\mathbb{P}^{2}$ pass through the point $[1:0:0]$. 
    Generally conics in $\mathbb{P}^{2}$ can be represented by degree 2 polynomials:
    \begin{align*}
        a_{0}x_{0}^{2}+a_{1}x_{1}^{2}+ a_{2}x_{2}^{2} + a_{3}x_{0}x_{1}+ a_{4}x_{0}x_{2}+a_{5}x_{1}x_{2}=0
    \end{align*}
    if we consider the parameter space $[a_{0}:a_{1}:a_{2}:a_{3}:a_{4}:a_{5}]$, this is just $P^{5}$.
    if the conics pass through $[1:0:0]$
    \begin{align*}
        a_{0}=0
    \end{align*}
    And by definition ${a_{0}= 0}$ is a submanifold of $P^{5}$.

\end{eg}

What if we have more than one conditions? Apparently, this means there are multiple submanifolds and 
what we need to consider is the intersection of these submanifolds. Then we changed the enumeration problem
to a intersection problem. To get a integer result, the dimension of the intersection has to be $0$, 
so we need an algebra to calculate the the "number"(which make sense with $0$ dimension)of intersection while keeping
track of the dimension. 

Another problem arised,which is more primary, is that what kind of intersection we are considering.
For example, a line can intersect with a surface in three ways, pass through the surface , tangent to the surface or contained
in the surface. Which one is more general than the other? Obviously, the first case. We can see this by changing the
position of the line in a small amount. The fisrt case remain the same. If we move the line in the diretion normal
to the surface, the tangent line pass through the surface. The result is similar for the third case.

And we will solve this two problems in this chapter.  Firstly, we revise the definition
of homology of a topological space.
\begin{df}
    Let $X$ be a topological space. We define the group of sigular k-chains $C_{k}(X)$
    to be the free abelian group generated by the set of all continuous maps from
    the standard k-simmplex to $X$:
    \begin{align*}
        C_{k}(X)=  \{\sum_{i}n_{i}f_{i}|n_{i} \in \mathbb{Z}, f_{i}\Delta_{k} \rightarrow X\}
    \end{align*}
\end{df}
As we known, each standard k-simplex has $k+1$ faces, each of which is homeomorphic to the standard
$(k-1)$-simplex. We use $j_{i,k}: \Delta_{k-1} \hookrightarrow \Delta_{k}^{i}$ to denote this inclusion map
.

\begin{df}
    \begin{align*}
         \partial_{k} : & C_{k}(X) \rightarrow C_{k-1}(X) \\
      & \sum_{i}n_{i}f_{i} \mapsto  \sum_{i,r}(-1)^{r}n_{i}(f_{i}\circ j_{r,k})
    \end{align*}
        
\end{df}

\begin{lm}
   
    $\partial_{k-1}\circ \partial_{k}=0$
    
    
\end{lm}

And now we have $Im(\partial_{k+1}) \subset Ker(\partial_{k})$. The group
$Z_{k}(X) = Ker(\partial_{k})$ is called the group of k-cycles and the group
$B_{K}(X) = Im(\partial_{k+1})$ is the group of k-boundaries.

\begin{df}
    The $k^{th}$ homology group is the quotient
    \begin{align*}
        H_{K}(X) = \frac{Z_{k}(X)}{B_{k}(X)}
    \end{align*}
\end{df}


Now we want to associate k-dimensional submanifolds of $X$ to homology classes in the $k^{th}$
homology group. So we can keep track of the dimension of different submanifolds. To make things 
easier, wo only consider the case when $X$ is an oriented manifold and only the oriented submanifolds
in $X$.

\begin{df}
    An oriented triangulation of $Z$ is a decomposition $Z= \cup Z_{i}$ of $Z$ into a union of
    finitedly many subsets $Z_{i}$ such that the following hold:
    \begin{align*}
        \text{There are orientation-preserving homeomorphisms} \\
        f_{i} : \Delta_{k} \rightarrow Z_{i} \\
        \text{Any intersection $Z_{i_{1}}\cap \ddots \cap Z_{i_{R}}$ is identified via any of the 
        $f_{i_{l}}$ with a face of $\Delta_{k}$
        }
    \end{align*}

\end{df}

Now we want to show if $Z$ is compact and oriented submanifold of a compact complex manifold
, $[Z]$ is a homology class. We choose a triangulation as above $T = \sum_{i} f_{i} \in C_{k}(X)$. 

\begin{lm}
    $\partial_{k}T = 0$
\end{lm}

\begin{proof}
    $\partial_{k}T=\sum_{i}g_{i}, g_{i}\in C_{k-1}(X) $.  As $Z$ compact and oriented, each of the 
    $(K-1)$-chain is in the boundaries of exactly two $k$-chains on the opposite side. So each of the
    $(k-1)$-chain appears in the sum exactly twice, once with a plus sign and once with a minus sign.
    So $\partial_{k}T = 0$.
\end{proof}

And this is true even when we choose a different triangulation.

However, homology is not the best tool to describe the intersection of these submanifolds.
Cohomology is the better tool and it is highly realated to homology via Poicare duality. We can
define it by a lot of definition we used for homology.

\begin{df}
    The group of $k$-cochains is 
    \begin{align*}
        C^{k}(X) = Hom (C_{k}(X),Z)
    \end{align*}
    The coboundary map is:
    \begin{align*}
        \delta_{k} : & C^{k}(X) \rightarrow C^{k+1}(X) \\
                     & (\delta_{k}(f))(Z) = f(\partial_{k+1}(Z))
    \end{align*}
    where $f \in C^{k}(X)$ and $Z \in C_{k+1}(X)$
\end{df}

\begin{lm}
    $\delta_{k}\circ \delta{k-1} = 0$
\end{lm}

\begin{proof}
    \begin{align*}
        ((\delta_{k}\circ \delta_{k-1})(f))(Z) &= (\delta_{k}(\delta_{k-1}(f)))(Z) \\
                                               &= (\delta_{k-1}(f)(\partial_{k}(Z)))\\
                                               &= f(\partial_{k-1}\circ \partial_{k}(Z) )= 0
    \end{align*}
    
\end{proof}
Similarly the group Ker$(\delta_{k})$ is called the group of k-cocycles $Z^{k}(X)$and the 
group Im$(\delta_{k})$ is called the group of k-cobundaries $B^{k}(X)$. and we have $B^{k}(X) \subset Z^{k}(X)$.

\begin{df}
    The $k^{th}$ cohomology group is the quotient 
    \begin{align*}
        H^{k}(X) = \frac{Z^{k}(X)}{B^{k}(X)}
    \end{align*}  
    And immediately there is a pairing between cohomology and homology:
    \begin{align*}
        \langle \; ,\ \rangle : & H^{k}(X)\times H_{k}(X) \rightarrow \mathbb{Z} \\
                                & \langle [f],[Z] \rangle = f(Z)         
    \end{align*}  

\end{df}


%%%%%%%%%%%%%%%%%%%%%%%%%%%%%%%%%%%%%%%%%%%%%%%%%%%%%%%%%%%%%%%%%%%%%%%%%%%%%%
\newpage
\section{Enumerative problem with lines}
%%%%%%%%%%%%%%%%%%%%%%%%%%%%%%%%%%%%%%%%%%%%%%%%%%%%%%%%%%%%%%%%%%%%%%%%%%%%%%
\newpage
\section{Vector bundles and chern class}

%%%%%%%%%%%%%%%%%%%%%%%%%%%%%%%%%%%%%%%%%%%%%%%%%%%%%%%%%%%%%%%%%%%%%%%%%%%%%%
\newpage
\section{Applications}
\end{document}















